\documentclass[a4paper,11pt]{article}
\usepackage[utf8]{inputenc}
\usepackage{polski}
\usepackage{graphicx}
\usepackage[margin=25mm]{geometry}
\usepackage{url}
\usepackage{float}


\newcommand{\pictureframe}[2]{
	\begin{figure}[H]
		\begin{center}
			\includegraphics[width=#1]{#2}
		\end{center}
	\end{figure}
}

\title{
Założenia wstępne projektu\\
\small{Rozproszone i obiektowe systemy baz danych}
}
\author{Jakub Dymon, Paweł Nowacki, Paweł Sulżycki}
\date{\today}

\begin{document}
\maketitle
\section{Temat i cel projektu}
Temat: "Aplikacja do zarządzania kadrami wykorzystująca rozproszoną bazę danych"\\
Cel projektu: projekt oraz implementacja aplikacji internetowej dla działu kadr w firmie branży IT
\section{Opis działania i funkcje systemu}
System pozwoli na dodawanie i usuwanie pracowników, przeglądanie i edycję danych osobowych pracowników i zależności hierarchicznych pomiędzy nimi. Ponadto system replikacji bazy danych pozwoli na obsługę systemu z różnych oddziałów firmy rozsianych po całym świecie.\\
System będzie miał formę aplikacji internetowej z systemem logowania i będzie dostępny dla pracowników działu kadr. Konta użytkowników będą zakładane odgórnie przez administratora.
\section{Założenia architektoniczne przyjęte podczas realizacji systemu}
Aplikacja będzie wykorzystywała relacyjną bazę danych MySQL oraz logikę biznesową zaimplementowaną z użyciem języka Java i biblioteki Spring Framework oraz pochodnych. Każda instancja aplikacji będzie posiadać własną bazę danych o takich samej zawartości i będzie dobierana na podstawie lokalizacji. Zastosowana będzie replikacja  migawkowa typu multimaster na poziomie bazy danych i będzie służyła propagacji danych pomiędzy różnymi instancjami tej samej aplikacji. Kontroli wersji danych będzie służyć mechanizm \textit{optimistic locking}. Instancja aplikacji, z którą łączyć się będzie użytkownik, będzie dobierana na podstawie lokalizacji użytkownika.
\section{Wykorzystywane narzędzia, technologie projektowania oraz implementacji systemu}
\begin{itemize}
	\item Java 8, Spring Framework
	\item MySQL 5.7.10
	\item HTML 5
	\item CSS 3
	\item \LaTeX
	\item Visual Paradigm
	\item IntelliJ IDEA
\end{itemize}
\section{Schemat komunikacji, struktura systemu}
\pictureframe{15cm}{Schemat.png}
\section{Literatura}
\begin{enumerate}
	\item \url{https://dev.mysql.com/doc/refman/5.7/en/mysql-cluster-replication-multi-master.html}
	\item \url{https://docs.spring.io/spring/docs/5.0.1.RELEASE/spring-framework-reference/core.html#spring-core}
	\item \url{http://docs.spring.io/spring-data/jpa/docs/1.10.4.RELEASE/reference/html/}
\end{enumerate}
\end{document}